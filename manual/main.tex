\documentclass{article}
\usepackage[utf8]{inputenc}
\pagenumbering{arabic}
\usepackage{ragged2e}

\usepackage{graphicx}
\usepackage{subfig}
\captionsetup[subfigure]{justification=centering,labelfont={bf,sf, scriptsize},textfont={bf,sf,scriptsize},singlelinecheck=off,justification=centering}
\captionsetup[figure]{justification=centering,labelfont={bf,sf, footnotesize},textfont={bf,sf,footnotesize},singlelinecheck=off}

\usepackage{wrapfig}
\usepackage{placeins}
\usepackage[brazil]{babel}
\usepackage{hyperref}




\title{\huge Manual de Uso - Spotify 9.1.2}
\author{Aline Daffiny Ferreira Gomes}
\date{08 de Dezembro de 2025}

\begin{document}

\maketitle

\tableofcontents

\newpage

\section{Introdução ao Sistema}

    \subsection{O que é o Spotify?}
    \justifying
    É uma plataforma de \emph{streaming} de áudio, que reúne músicas e podcasts num único ambiente. Com o Spotify, é possível acompanhar seus artistas favoritos, criar \emph{playlists}, explorar novas músicas e receber recomendações baseadas no que o usário consome. Disponível para dispositivos móveis, computadores, \emph{Smart TVs} e muito mais, oferece a praticidade de poder acessar o sistema e ouvir suas músicas em praticamente qualquer lugar. \par
    O Spotify oferece um plano gratuito, com restrições na experiência, e para acessar todas as funcionalidades, é necessário adquirir a assinatura do modo Premium, que permite baixar músicas para ouvir de forma \emph{offline}, maior qualidade de áudio, experiência sem anúncios, entre outros benefícios.

    \subsection{Qual problema o Spotify resolve?}
    \justifying
    Com o avanço da tecnologia, e a migração da mídia física para  ambiente digital, o Spotify surge como uma solução segura, legalizada e centralizada, sem a necessidade de armazenar as faixas de áudio ou se preocupar com a instalação de vírus. \par 
    O serviço também soluciona questões de mobilidade e praticidade, oferecendo acesso a milhões de músicas por um único preço, possibilidade de ouvir músicas de forma \emph{offline} e um catálogo constantemente atualizado, garantindo que lançamentos estejam disponíveis imediatamente.

    \subsection{Quem é o público-alvo da plataforma?}
    Desde ouvintes casuais, dispostos a descobrir novos artistas, até fãs assíduos, que buscam estar atualizados com seus ídolos. De forma geral, pessoas com o hábito de incluir uma trilha sonora na rotina, que prezem por uma experiência simples e acessível. 

\newpage


\section{Acesso ao Sistema}

    \subsection{Pré-requisitos}
    \begin{itemize}
        \item Possuir, no mínimo, 13 anos de idade;
        \item Possuir um dispositivo compatível com o aplicativo, como aparelho Android, IOS, Windows, Mac, entre outros;
        \item No caso de optar pelo aplicativo, possuir espaço disponível suficiente para instalação do app;
        \item Possuir um e-mail para criação da conta;
        \item Acesso à Internet;
        \item Para realização da assinatura, possuir um método de pagamento válido, como cartão de crédito, débito, PayPal, entre outros.
    \end{itemize}

    
    \subsection{Como criar uma conta no Spotify?}
    \subsubsection{Navegador}
    \label{section:account_navegador}
        \begin{enumerate}
            \item Acesse o site em \href{URL}{https://open.spotify.com/intl-pt};
            
            \item No canto superior direito, clique em "Inscrever-se";
            \begin{figure}[h]
                \centering
                \includegraphics[width=0.8\linewidth]{account_web/web01.png}
                \caption{Localização do botão "Inscrever-se" na página principal}
                \label{fig:web:01}
            \end{figure}

            \item Na página de inscrição, digite um e-mail válido no campo fornecido. Em seguida, clique em "avançar";

            \item Crie uma senha para sua conta, cumprindo com os requisitos exigidos listados na tela, e clique em "avançar";

            \item Digite seu nome e data de nascimento em seus respectivos campos, e marque a opção com a identidade de gênero você se identifica. Clique em "avançar";

            \item Finalize a inscrição marcando as caixas de Termos e Condições, e clique em "Inscrever-se";

            \item Você será redirecionado novamente para a página principal. Dessa vez, clique em "Entrar", no canto superior direito;
            \begin{figure}[h!]
                \centering
                \includegraphics[width=0.8\linewidth]{account_web/web02.png}
                \caption{Localização do botão "Entrar" na página principal}
                \label{fig:web:02}
            \end{figure}

            \item Digite o e-mail utilizado anteriormente para criar a conta e clique em "continuar";

            \item Um código será enviado para o e-mail fornecido. Digite-o no campo indicado;

            \item \textbf{Pronto!} Sua conta Spotify está pronta para ser utilizada.
        \end{enumerate}
    \subsubsection{Aplicativo para Dispositivos Móveis}
        \begin{enumerate}
            \item Baixe o aplicativo do Spotify na sua loja de aplicativos nativa;
            \begin{figure}[h!]
                \centering
                \subfloat[Play Store]{\includegraphics[width=0.3\linewidth]{account_mobile/mobile1_1.png}} \quad
                \subfloat[App Store]{\includegraphics[width=0.3\linewidth]{account_mobile/mobile1_2.png}}
                \caption{Página de download do app nos principais sistemas operacionais mobile}
                \label{fig:mobile:01}
            \end{figure}

            \item Abra o aplicativo, e clique no botão verde "Inscreva-se grátis" na parte inferior da tela. Serão fornecidas diferentes opções para prossegir: e-mail, número de telefone e Google. Continue por onde preferir, o tutorial irá seguir com a opção de e-mail;

            \item Digite seu e-mail no campo fornecido e clique em "avançar";

            \item Crie uma senha para sua conta e clique em "avançar";

            \item Digite sua data de nascimento e clique em "avançar"

            \item Digite o seu nome e marque as opções de Termos e Condições que você concorda;

            \item O aplicativo irá solicitar para você ativar as notificações. Caso deseje ativar, clique em "Ativar notificações" e habilite a função, do contrário, clique em "Agora não";

            \item Uma lista artistas será apresentada. Selecione pelo menos 3 artistas que você gosta. Caso não encontre seu artista preferido, você pode buscar pelo nome dele no campo de busca na partre superior da tela. Essa etapa serve para personalizar a sua experiência;
            \begin{figure}[h!]
                \centering
                \includegraphics[width=0.3\linewidth]{account_mobile/mobile02.png}
                \caption{Selecione artistas que você gosta para personalizar sua experiência}
                \label{fig:mobile02}
            \end{figure}

            \item Em seguida, selecione podcasts do seu interesse. Essa etapa é opcional. Quando concluído, clique em "concluído" no botão branco na parte inferior da tela;

            \item O Spotify irá criar uma playlist personalizada baseada nos gostos que você acabou de listar. Você pode ouvir clicando no botão "Começar a ouvir", ou não ouvir e clicar em "Agora não";

            \item \textbf{Pronto!} Sua conta está pronta para ser utilizada.
        \end{enumerate}
    \subsubsection{Aplicativo para Computadores}
        \begin{enumerate}
            \item Baixe o aplicativo do Spotify pelo navegador, ou pelas lojas oficias do seu sistema operacional;
            \begin{figure}[h!]
                \centering
                \subfloat[Navegador]{\includegraphics[width=0.3\linewidth]{account_pc/pc1_1.png}} \quad
                \subfloat[Microsoft Store]{\includegraphics[width=0.3\linewidth]{account_pc/pc1_2.png}} \quad
                \subfloat[App Store]{\includegraphics[width=0.3\linewidth]{account_pc/pc1_3.png}}
                \caption{Página de download do aplicativo no computador. Localização da página de download na figura a}
                \label{fig:pc01}
            \end{figure}

            \item Abra o aplicativo e selecione a opção "Inscreva-se grátis";
            \begin{figure}[h!]
                \centering
                \includegraphics[width=0.8\linewidth]{account_pc/pc02.png}
                \caption{Localização da opção para criar conta}
                \label{fig:pc02}
            \end{figure}

            \item Você será encaminhado para a página de criação de conta no navegador do seu aparelho. A partir desta etapa, os passos são o mesmo da \textbf{seção \ref{section:account_navegador}}.
        \end{enumerate}
        
\FloatBarrier
    \justifying
    Outros dispositivos, como Smart TVs, Smartwatches e Telas multimídia de veículos, terão o método de \emph{login} integrados com o aplicativo ou com o navegador, portanto, os passos para criação de conta são os mesmos.
        
\newpage

\section{Visão Geral da Interface}
\justifying
Essa seção apresenta as principais funções e menus acessíveis a partir da página principal do sistema. Funcionalidades mais específicas serão apresentadas a explicadas na próxima seção: \textbf{\ref{section:funcoes} Principais Funcionalidades}.

    \subsection{Aplicativo para Dispositivos Móveis}
    \justifying
    Ao entrar no aplicativo, o usuário se encontra na página principal, na aba "Início", como indicado na parte inferior \textbf{(4)} pelo símbolo em destaque. \par 
    
    \begin{wrapfigure}{l}{6cm}
    \centering
    \includegraphics[width=0.8\linewidth]{interface_mobile/mobile_interface.png}
    \caption{Interface do aplicativo para dispositivos móveis}
    \label{fig:interface_mob01}
    \end{wrapfigure}

    \justifying
    Nessa tela, é possível acessar o menu lateral clicando no canto superior esquerdo \textbf{(1)}, ou deslizando a tela da lateral esquerda para o centro. Para voltar à tela principal, basta deslizar o menu para a esquerda ou clicar fora do menu, na tela escurecida. Ao lado da foto de perfil, existem filtros de conteúdo \textbf{(2)}, para mostrar todo o conteúdo recomendado disponível, somente músicas, somente podcasts, ou o modo "Retrospectiva", disponível uma vez por ano, no fim do ano, com recomendações e estatísticas do que o usuário consumiu durante o ano. \par
    Logo abaixo, ainda na porção superior da tela, é apresentado um breve histórico \textbf{(3)} das últimas playlists, álbuns ou artistas que o usuário escutou, fornecendo acesso rápido. Na parte inferior existe um pequeno menu de navegação \textbf{(4)} da página principal, possibilitando acesso para a área inicial, área de busca, biblioteca do usuário, e um atalho rápido para criação. As seções da página principal são acessíveis tanto pelos botões no menu inferior, quanto por gestos deslizantes para a esquerda ou para a direita. \par
    A porção entre a área \textbf{(3)} e \textbf{(4)} é um \emph{feed} de navegação, com recomendações personalizadas baseadas nos gostos do usuário, podendo ser explorada com gestos deslizantes para cima ou para baixo. \par

    \justifying
    Clicando no símbolo de lupa, no menu inferior, o usuário segue para a área de busca (Figura \ref{fig:interface_mob02}), tendo a opção de escrever \textbf{(1)} o nome de uma música, álbum, artista ou trecho da letra, para iniciar a busca. Caso o usuário possua um código Spotify (Figura \ref{fig:codes}), o mesmo pode ser lido clicando no ícone de câmera \textbf{(2)} no canto superior direito. Na área \textbf{(3)} existem recomendações rápidas no formato de vídeos curtos, navegando por músicas com \emph{tags} semelhantes as que o usário escuta. \par.
    
    \begin{wrapfigure}{r}{4cm}
    \centering
    \includegraphics[width=0.8\linewidth]{interface_mobile/mobile_busca.png}
    \caption{Área de busca na página principal}
    \label{fig:interface_mob02}
    \end{wrapfigure}

    \justifying
    Por fim, na região inferior \textbf{(4)}, o aplicativo fornece diversos filtros para buscar faixas em uma categoria específica. Clicando em "Sua biblioteca", no menu inferior, o usuário poderá encontrar todos os itens que favoritou, desde músicas, playlists, álbuns, artistas e podcasts (Figura \ref{fig:interface_mob03}). \par
    No canto superior direito \textbf{(1)} existem dois ícones: uma lupa, que permite a busca de itens especificamente na biblioteca do usuário, e um símbolo de soma, que abre o menu de criação, que será mencionado mais profundamente em breve. Abaixo, alguns filtros de conteúdo \textbf{(2)}, e, abaixo dos filtros, funções para organizar a visualização da biblioteca. Os itens podem ser organizados por ordem de visualização, adição à biblioteca, ordem alfabética pelo nome do item, ou ordem alfabética pelo nome do criador do item. A visualização pode ser em formato de lista ou de grade. \par
    O último ícone do menu inferior leva para o menu de criação, também acessível no canto superior direito da biblioteca. \par

    \begin{wrapfigure}{l}{4cm}
        \centering
        \includegraphics[width=0.8\linewidth]{interface_mobile/mobile_biblioteca.png}
        \caption{Área da biblioteca na página principal}
        \label{fig:interface_mob03}
    \end{wrapfigure}

    \justifying
    O menu de criação (Figura \ref{fig:interface_mob04}) permite ao usuário criar playlists; playlists colaborativas, onde todos os participantes podem adicionar músicas; um \emph{Match}, que é uma playlist criada pelo Spotify, combinando o gosto musical de dois usuários; e uma \emph{Jam}, que é uma espécie de playlist colaborativa que permite que vários usuários escutem simultâneamente às mesmas faixas, independente da distância. Para sair do menu de criação basta clicar em qualquer lugar fora do menu, na tela escurecida. \par
    A interface para Smart TVs e telas multimídia veiculares se assemelha à interface \emph{mobile}, com pequenas mudanças de ícones e ausência ou inclusão de novas áreas na página principal, mas a navegação se torna intuitiva, uma vez que cada ícone possui uma legenda indicando sua função. A interface para computadores, entretanto, seja no modo navegador ou aplicativo, se diferencia mais da interface das outras plataformas, e, portanto, será descrita na próxima subseção. \par

    \FloatBarrier
        
    \begin{figure}[p]
        \centering
        \includegraphics[width=0.5\linewidth]{interface_mobile/spotify_codes.png}
        \caption{Exemplo de códigos Spotify}
        \label{fig:codes}
    \end{figure}

    \begin{figure}[h]
        \centering
        \includegraphics[width=0.3\linewidth]{interface_mobile/mobile_criar.png}
        \caption{Menu de criação}
        \label{fig:interface_mob04}
    \end{figure}

\FloatBarrier
\newpage
    
    \subsection{Navegador/Aplicativo para Computadores}

    Assim que o aplicativo ou site é aberto, e o usuário entra com a conta, essa é a interface apresentada. \par

    \begin{figure}[h]
        \centering
        \includegraphics[width=0.8\linewidth]{interface_pc/interface_pc.png}
        \caption{Interface do Spotify para navegador ou aplicativo desktop}
        \label{fig:interface_pc01}
    \end{figure}

    \justifying
    Nessa interface, não é necessário ir para uma área específica para iniciar uma busca, a barra de busca \textbf{(1)} se localiza no topo da página, juntamente com um ícone de casa, à esquerda, para voltar para a área inicial, e um ícone de gaveta, à direita, para navegar por categorias.  Abaixo, os filtros de conteúdo \textbf{(2)} e o histórico recente \textbf{(3)}, de forma similar a interface \emph{mobile}. \par
    Assim como a área de busca, a biblioteca não possui uma área para si, no lugar, existe uma barra lateral expansível, à esquerda,\textbf{(4)} com a capa dos itens salvos. Para expandir a biblioteca, e ler os nomes, basta clicar no ícone no topo da barra lateral. Á direita, uma janela com informações sobre a faixa tocando. Na parte inferior \textbf{(6)}, o \emph{player} de áudio, com nome, duração, e botões de ação para favoritar, pausar, pular a faixa, aumentar o volume, entre outros. \par
    Por último, no canto superior direito, um ícone de sino, para acessar notificações; um ícone de usuários para ver a atividade dos seus amigos; e sua foto de perfil que, quando clicada, dá acesso ao menu lateral, possibilitando acesso a algumas configurações.

    
\FloatBarrier
\newpage

\section{Principais Funcionalidades}
\label{section:funcoes}

\justifying
O Spotify oferece diversas funcionalidades, incluindo recursos sociais, ferramentas de monitoramento e integrações internas, e, uma vez acostumado com a interface, se tornam intuitivas para descobrir. Nessa seção, descrevemos as principais e mais importantes funcionalidades, que incluem:
    \begin{itemize}
        \item Criar uma playlist
        \item Encontrar uma faixa
        \item Ouvir a faixa
        \item Ver a letra de uma música
        \item Salvar a faixa
        \item Seguir um artista
        \item Realizar assinatura do Premium
        \item Baixar músicas para ouvir offline
        \item Iniciar uma Jam
        \item Verificar suas estatísticas
    \end{itemize}
Exclusivamente nessa seção, será considerado apenas a versão do sistema para dispositivos móveis.

    \newpage
    
    \subsection{Criar uma playlist}
    \begin{enumerate}
        \item A partir da tela principal, clique no ícone para criar, no canto inferior direito;
        \item Selecione a opção "Playlist";
        \item Dê um nome para sua playlist e clique em "criar";

        \begin{figure}[h!]
            \centering
            \includegraphics[width=0.4\linewidth]{funcionalidades/playlist.png}
            \caption{Playlist personalizada}
            \label{fig:playlist}
        \end{figure}
        
        \item A playlist é, por padrão, de visibilidade pública. Caso deseje que os outros usuários não tenham acesso à sua playlist, clique nos 3 pontinhos \textbf{(1)}, embaixo da capa da playlist e selecione "Tornar playlist particular";
        \item O próprio Spotify irá sugerir algumas músicas \textbf{(2)} para adicionar na playlist, mas essa etapa é opcional.
    \end{enumerate}
    
    \subsection{Encontrar uma faixa}
    \begin{enumerate}
        \item Clique no ícone de lupa e vá para a área de busca;
        \item Neste exemplo, a busca é feita utilizando um trecho de música, mas você pode buscar por um título ou nome de artista;
        
        \begin{figure}[h!]
            \centering
            \includegraphics[width=0.4\linewidth]{funcionalidades/busca.png}
            \caption{Resultados de busca por trecho de letra}
            \label{fig:busca}
        \end{figure}
        
        \item O Spotify indica os resultados que estão sendo recomendados com base na letra \textbf{(1)};
        \item Os resultados podem incluir o artista da música que você procura \textbf{(2)};
        \item Também podem incluir músicas, álbuns e playlists com o mesmo nome \textbf{(3)};
        \item ou Playlists que contém a música pesquisada.
    \end{enumerate}

    \newpage
    
    \subsection{Ouvir a faixa}
    \begin{enumerate}
        \item A partir de uma busca ou estando em uma playlist, para ouvir uma faixa basta clicar sobre o item desejado
        \item O player de áudio será exibido, com botão de pause e botões para voltar ou avançar uma música (uso limitado para contas gratuitas).
        
        \begin{figure}[h!]
        \centering
        \includegraphics[width=0.4\linewidth]{funcionalidades/player.jpeg}
        \caption{Interface do player de áudio}
        \label{fig:player}
        \end{figure}
    \end{enumerate}

        
    \subsection{Ver a letra de uma música}
    \begin{enumerate}
        \item Dentro do \emph{player} de áudio, deslize levemente a tela da parte inferior para a parte superior;
        \item Caso a letra esteja disponível para essa faixa, você verá uma janela com a letra;
        \item Para navegar pela letra completa ou expandir a mesma, basta clicar sobre a janela ou no ícone de setas opostas no canto superior direito da janela;
        \item Algumas músicas possuem sincronia com a letra, possibilitando acompanhar em tempo real enquanto escuta, sem precisar descer manualmente.
        \item Para retornar ao player, clique na seta apontada para baixo, no canto superior esquerdo da tela.

        \begin{figure}
            \centering
            \subfloat[Janela com letra]{\includegraphics[width=0.4\linewidth]{funcionalidades/letra_janela.png}} \quad
            \subfloat[Letra expandida]{\includegraphics[width=0.4\linewidth]{funcionalidades/letra_tela.png}}
            \caption{Telas com a letra da faixa}
            \label{fig:letra}
        \end{figure}
        \end{enumerate}
        
    \subsection{Salvar a faixa}
    \begin{enumerate}
        \item Dentro do \emph{player} de áudio, clique no ícone de círculo com um símbolo de soma dentro, localizado ao lado do nome da faixa. É possível vê-lo na Figura \ref{fig:player};
        \item Uma notificação aparecerá na parte inferior da tela, confirmando que a música foi salva na sua playlist de músicas curtidas;        
        \item Caso deseje adicionar em outra playlist, clique em "Alterar" na notificação ou clique no ícone verde, lozalizado no mesmo lugar do ícone de adicionar à playlist;
        
        \begin{figure}[h]
            \centering
            \includegraphics[width=0.4\linewidth]{funcionalidades/alterar_playlist.png}
            \caption{Notificação ao clicar para adicionar a música em uma playlist}
            \label{fig:notif}
        \end{figure}
        
        \item Clique sobre a playlist que você deseja adicionar a música. Caso não encontre, use a barra de busca escrita "Encontrar playlist" (Figura \ref{fig:list} \textbf{(1)});

        \begin{figure}[h]
            \centering
            \includegraphics[width=0.4\linewidth]{funcionalidades/playlist_list.png}
            \caption{Lista de playlist da sua biblioteca}
            \label{fig:list}
        \end{figure}
        
        \item Quando tiver concluído, deslize a tela do topo para a parte inferior, até que a aba de playlists suma.
        \end{enumerate}

    \newpage

    \subsection{Seguir um artista}
    \begin{enumerate}
        \item Dentro de um \emph{player} de áudio, clique no nome do artista, localizado embaixo do nome da faixa. Você também pode pesquisar pelo nome do artista e clicar sobre o resultado;
        \item Você será direcionado para a página do artista
        \item Clique no botão "Seguir", abaixo do nome do artista.

        \begin{figure}[h]
            \centering
            \includegraphics[width=0.4\linewidth]{funcionalidades/follow.png}
            \caption{Página do artista}
            \label{fig:seguir}
        \end{figure}
    \end{enumerate}

    \newpage
    
    \subsection{Realizar assinatura do Premium}
    As próximas funções necessitam da assinatura Premium, portanto vamos ver como realizá-la.
    \begin{enumerate}
        \item Na página inicial, clique no ícone com a logo do Spotify, escrito "Premium", no menu inferior. Em contas gratuitas, ele aparece entre a bilbioteca e o menu de criação;

        \begin{figure}[h!]
            \centering
            \includegraphics[width=0.5\linewidth]{funcionalidades/premium_buttom.jpeg}
            \caption{Localização do botão "Premium" em contas gratuitas}
            \label{fig:premium}
        \end{figure}
        
        \item Escolha o plano que se encaixa no seu orçamento, e escolha entre manter uma assinatura (renovada automaticamente) ou pagar por um único mês;
        \item Você será encaminhado para a página de pagamento, onde poderá escolher como pagar.
    \end{enumerate}
    Uma vez que o pagamento for finalizado e aprovado, as próximas funções ficarão disponíveis.
        
    \subsection{Baixar músicas para ouvir offline}
    No momento, não é possível baixar uma única música, somente playlists ou álbuns.
    \begin{enumerate}
        \item Em uma playlist ou álbum, procure pelo ícone de círculo com uma seta para baixo dentro. Clique no ícone;
        \item O download irá iniciar, e o ícone anterior deverá aparecer verde quando concluído;
        
        \begin{figure}[h]
            \centering
            \subfloat[Antes do download]{\includegraphics[width=0.4\linewidth]{funcionalidades/download.png}} \quad
            \subfloat[Depois do download]{\includegraphics[width=0.4\linewidth]{funcionalidades/downloaded.png}}
            \caption{Comparação do ícone antes e depois do download}
            \label{fig:download_buttom}
        \end{figure}
        
        \item Próxima vez que estiver sem Internet e acessar o aplicativo, a música estará disponível para ouvir.
        
    \end{enumerate}
    \subsection{Iniciar uma Jam}
    \begin{enumerate}
        \item Clique no ícone de "Criar" no menu inferior;
        \item Selecione a opção "Jam";
        \item A Jam será criada, e uma janela com o link e o código QR para compartilhamento irá aparecer.

        \begin{figure}[h]
            \centering
            \includegraphics[width=0.4\linewidth]{funcionalidades/jam.jpeg}
            \caption{Link e código QR para compartilhamento da Jam}
            \label{fig:jam_share}
        \end{figure}
    \end{enumerate}

    \newpage
    
    \subsection{Verificar suas estatísticas}
    \begin{enumerate}
        \item Na página principal, clique na sua foto de perfil e abra o menu lateral;
        \item Selecione a opção "Sua cápsula sonora";

        \begin{figure}[h]
            \centering
            \includegraphics[width=0.4\linewidth]{funcionalidades/stats.png}
            \caption{Localização da Cápsula sonora}
            \label{fig:capsula}
        \end{figure}
        
        \item Você será direcionado para um feed com estatísticas mensais sobre o seu uso do aplicativo (Figura \ref{fig:stats}), incluindo tempo ouvido, artista e música mais ouvida.

        \begin{figure}
            \centering
            \includegraphics[width=0.4\linewidth]{funcionalidades/stats02.jpeg}
            \caption{Interface da Cápsula sonora}
            \label{fig:stats}
        \end{figure}
    \end{enumerate}


\end{document}
